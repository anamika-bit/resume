%%%%%%%%%%%%%%%%%
% This is an sample CV template created using altacv.cls
% (v1.1.4, 27 July 2018) written by LianTze Lim (liantze@gmail.com). Now compiles with pdfLaTeX, XeLaTeX and LuaLaTeX.
% 
%% It may be distributed and/or modified under the
%% conditions of the LaTeX Project Public License, either version 1.3
%% of this license or (at your option) any later version.
%% The latest version of this license is in
%%    http://www.latex-project.org/lppl.txt
%% and version 1.3 or later is part of all distributions of LaTeX
%% version 2003/12/01 or later.
%%%%%%%%%%%%%%%%
\newcommand{\RNum}[1]{\uppercase\expandafter{\romannumeral #1\relax}}
%% If you need to pass whatever options to xcolor
\PassOptionsToPackage{dvipsnames}{xcolor}

%% If you are using \orcid or academicons
%% icons, make sure you have the academicons 
%% option here, and compile with XeLaTeX
%% or LuaLaTeX.
% \documentclass[10pt,a4paper,academicons]{altacv}

%% Use the "normalphoto" option if you want a normal photo instead of cropped to a circle
% \documentclass[10pt,a4paper,normalphoto]{altacv}

\documentclass[10pt,a4paper]{altacv}
%% AltaCV uses the fontawesome and academicon fonts
%% and packages. 
%% See texdoc.net/pkg/fontawecome and http://texdoc.net/pkg/academicons for full list of symbols.
%% 
%% Compile with LuaLaTeX for best results. If you
%% want to use XeLaTeX, you may need to install
%% Academicons.ttf in your operating system's font 
%% folder.


% Change the page layout if you need to
\geometry{left=1cm,right=9cm,marginparwidth=6.8cm,marginparsep=1.2cm,top=1.25cm,bottom=1.25cm,footskip=2\baselineskip}

% Change the font if you want to.

% If using pdflatex:
\usepackage[T1]{fontenc}
\usepackage[utf8]{inputenc}
\usepackage[default]{lato}

% If using xelatex or lualatex:
% \setmainfont{Lato}

% Change the colours if you want to
\definecolor{Navy}{HTML}{000080}
\definecolor{SlateGrey}{HTML}{2E2E2E}
\definecolor{LightGrey}{HTML}{666666}
\colorlet{heading}{Navy}
\colorlet{accent}{Navy}
\colorlet{emphasis}{SlateGrey}
\colorlet{body}{LightGrey}

% Change the bullets for itemize and rating marker
% for \cvskill if you want to
\renewcommand{\itemmarker}{{\small\textbullet}}
\renewcommand{\ratingmarker}{\faCircle}
%% sample.bib contains your publications
\addbibresource{sample.bib}

\usepackage[colorlinks]{hyperref}

\begin{document}

\name{ANAMIKA CHOUDHARY}
\tagline{Keshav Mahavidyalaya, Delhi University, Delhi-India, 110034}
% \photo{2.8cm}{Globe_High}
\personalinfo{%
  % Not all of these are required!
  % You can add your own with \printinfo{symbol}{detail}
  \email{anamika.2661@gmail.com}
  \phone{(+91) 8595234612}
%  \mailaddress{Address, Street, 00000 County}
%  \location{Gwalior, India}
%  \homepage{abhinavj004.github.io}
%  \twitter{@marissamayer}
  \linkedin{https://www.linkedin.com/in/anamika-choudhary-171030194 }
  \newline
  % \github{github.com/abhinavj004} % I'm just making this up though.
  %% You MUST add the academicons option to \documentclass, then compile with LuaLaTeX or XeLaTeX, if you want to use \orcid or other academicons commands.
%   \orcid{orcid.org/0000-0000-0000-0000}
}

%% Make the header extend all the way to the right, if you want. 
\begin{fullwidth}
\makecvheader
\end{fullwidth}

%% Depending on your tastes, you may want to make fonts of itemize environments slightly smaller
% \AtBeginEnvironment{itemize}{\small}


%% Provide the file name containing the sidebar contents as an optional parameter to \cvsection.
%% You can always just use \marginpar{...} if you do
%% not need to align the top of the contents to any
%% \cvsection title in the "main" bar.
\cvsection[page1sidebar]{Experience}

\cvevent{Web Development Training}{Keshav Mahavidyalaya}{August 2020 -- October 2020}{New Delhi}
\begin{itemize}
\item Developed web pages,forms and time table.  
\item Technology used for backend:PHP,MySql.
\item Technology used for frontend:HTML,CSS,Bootstrap,Jquery,Javascript.
\end{itemize}
\divider
\medskip
%\cvevent{Python Development Intern}{Keshav Mahavidyalaya}{December 2020 }{New Delhi}
%\begin{itemize}
%\item Work as a python developer intern to develop python modules for machine learning algorithms. 
%\end{itemize}



\cvsection{Projects}

\cvevent{Create to-do app}{Self-Learning Project}{May 2020}{}
\begin{itemize}
\item React js, postgre sql, fomantic UI
\item Javascript , libraries like flask, SQl Alchemy
%\item Evaluating the captions using BLEU scores as the metric.

\end{itemize}
\medskip
\divider

\cvevent{Bluetooth controlled car}{Self-Learning Project}{Feb 2020 }{}
\begin{itemize}
   \item Arduino, C++,sensors
%\end{itemize}
%\divider
%\medskip

%\cvevent{Sentiment Analysis of Twitter Dataset }{Exploratory Project}{October 2018 -- November 2018}{}
%\begin{itemize}
 %  \item Two methods were adopted for building the classifer.In the first,words are vectorized using frequency count and RNN was used.
  %  \item In the second approach these word vectors were given weights according to POS polarity and Naive Bayesian Classifier was used.
%\end{itemize}
%\divider
%\medskip
%\cvevent{Spam Classification Application }{Exploratory Project}{January 2019 }{}
%\begin{itemize}
 %   \item Develop a Machine Learning based application with Flask and use it to predict and detect youtube spam comments.

    
%\end{itemize}
%\medskip
% Adapted from @Jake's answer from http://tex.stackexchange.com/a/82729/226
% \wheelchart{outer radius}{inner radius}{
% comma-separated list of value/text width/color/detail}


\clearpage




%% If the NEXT page doesn't start with a \cvsection but you'd
%% still like to add a sidebar, then use this command on THIS
%% page to add it. The optional argument lets you pull up the 
%% sidebar a bit so that it looks aligned with the top of the
%% main column.
% \addnextpagesidebar[-1ex]{page3sidebar}

\end{document}
